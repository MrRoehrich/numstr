\chapter*{Abstrakt}
\flushleft
Der vorliegende Praktikumsbericht gibt einen \"Uberblick \"uber die im Fach Numerische Str\"omungssimulation erstellten Str\"omungsl\"oser der inkompressiblen Navier-Stokes-Gleichungen. \\ [0.5cm]

Das Dokument ist in mehrere einzelne Projekte unterteilt, die sich schrittweise zu den endg\"ultigen L\"osern vervollst\"andigen. In diesem Bericht werden die Modellierung der str\"omungsmechanischen Probleme, deren resultierende mathematische Beschreibung und die numerischen Verfahren zur Approximation der L\"osung besprochen. Besonderes Augenmerk wird im Weiteren auf  die Validierung der implementierten Algorithmen gelegt, indem die vom Computer gelieferten Ergebnisse f\"ur einfache Probleme mit deren analytischen L\"osungen verglichen werden. Desweiteren werden im letzten Kapitel Anwendungsf\"alle dargestellt, bei denen die Ergebnisse der Approximationen f\"ur komplexere, reale Problemstellungen diskutiert werden. \\[1cm]

Als Erstes werden zweidimensionale, inkompressible, reibungs- und wirbelfreie Str\"omungen mithilfe der Potentialtheorie bearbeitet und die L\"osung der resultierenden partiellen Differentialgleichung durch Diskretisierung mittels Finiter Differenzen approximiert. Hierbei werden zuerst kartesische und anschlie\ss{}end krummlinig berandete Integrationsgebiete verwendet. \\ [0.5cm]

Im zweiten Teil des Programmierpraktikums werden sodann auch reibungsbehaftete Str\"omungen betrachtet. Ein auf der Finite-Volumen-Methode basierender L\"oser verwendet zuerst gegebene Geschwindigkeiten, errechnet dann das zu einem gegebenen Druckfeld passende Geschwindigkeitsfeld, um schlussendlich sowohl Druck- als auch Geschwindigkeitsverteilung selbststs\"andig zu bestimmen. Auf der Grundlage dieses Str\"omungszustandes erfolgt der Transport eines passiven Skalars.
