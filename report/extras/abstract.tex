\chapter*{Abstract}
\flushleft
\textcolor{red}{BLOCKSATZ EINSTELLEN!}
Der vorliegende Praktikumsbericht gibt einen \"Uberblick \"uber den im Fach \texttt{Numerische Str\"omungssimulation} erstellten Str\"omungsl\"oser f\"ur die inkompressiblen Navier-Stokes-Gleichungen. \\ [0.5cm]

Das Dokument ist in mehrere einzelne Projekte unterteilt, die sich schrittweise zu dem endg\"ultigen L\"oser vervollst\"andigen. In diesem Bericht werden die Modellierung des str\"omungsmechanischen Problems, dessen mathematische Beschreibung und die numerischen Verfahren zur Approximation der L\"osung besprochen. Besonderer Augenmerk wird auf  die Validierung der implementierten Algorithmen gelegt, indem die vom Computer gelieferten Ergebnisse f\"ur einfache Probleme mit deren analytischen L\"osungen verglichen wurden. Des Weiteren wird je ein Anwendungsfall dargestellt, wo die Ergebnisse der Approximation f\"ur schwierigere, reale Probleme diskutiert werden. \\[1cm]

Als Erstes wird ein potentialtheoretisches Str\"omungsproblem mithilfe von Finite-Differenzen-Verfahren gel\"ost, sowohl in einem kartesischen Gitter, als auch in einem gekr\"ummten Kanal. Anschlie\ss{}end werden Finite-Volumen-Verfahren erwendet, um auch kompliziertere Gleichungen, welche mitunter Reibung und Inkompressibilit\"at modellieren zu l\"osen bzw. zu approximieren. Dabei wird zun\"achst ein Str\"omungsfeld vorgegeben, sp\"ater wird ebenfalls das Geschwindigkeitsfeld bei Vorgabe des Druckes berechnet, am Ende wird auch noch das Druckfeld numerisch bestimmt.
