\chapter{Berechnung des Geschwindigkeitsfeldes}

\section{Aufgabenstellung}
In der bisherigen Implementierung der Konvektion-Diffusion-Str�mung ist das Geschwindigkeitsfeld als gegeben angenommen worden. Nun soll dieses Geschwindigkeitsfeld unter Vorgabe des Druckfeldes w�hrend der Berechnung berechnet werden. Im Folgenden wird auf die mathematische Herleitung der ben�tigten Formeln eingegangen und anschlie�end die Software mit einfachen Testf�llen, wie beispielsweise die Couette-Str�mung, getestet.

\section{Mathematische Modellbildung}
F�r die Berechnung der beiden Geschwindigkeitskomponenten $u$ und $v$ wird eine weitere (vektorielle) Formel ben�tigt. Zun�chst wird die Impulsgleichung in x-Richtung
\begin{equation} \label{eq:impulsglg_2D}
\dfrac{\partial\left(\rho u\right)}{\partial t}+\dfrac{\partial\left(\rho u\cdot u\right)}{\partial x}+\dfrac{\partial\left(\rho v\cdot u\right)}{\partial y}=-\dfrac{\partial p}{\partial x}+\eta\dfrac{\partial^{2}u}{\partial x^{2}}+\eta\dfrac{\partial^{2}u}{\partial y^{2}}
\end{equation}
betrachtet, bzw. im eindimensionalen Fall
\begin{equation} \label{eq:impulsglg_1D}
\dfrac{\partial\left(\rho u\right)}{\partial t}+\dfrac{\partial\left(\rho u\cdot u\right)}{\partial x}=-\dfrac{\partial p}{\partial x}+\eta\dfrac{\partial^{2}u}{\partial x^{2}}.
\end{equation}

\section{Diskretisierung}
F�r die Diskretisierung wird ein versetzes Gitter verwendet. Da bei der Diskretisierung auf dem Gitter f�r den Skalar $\Phi$ die Druckwerte benachbarter Gitterzellen entkoppelt werden w�rden, w�rde beispielsweise ein oszillierender Druck (Zick-Zack-Verteilung), welcher von einer zur anderen Zelle jeweils von Wert 1 auf Wert 2 oder umgekehrt springt, entkoppelt werden. Ddass hei�t das bei der Betrachtung von Zelle $i$ nur die Druckwerte in den Zellen $i-1$ und $i+1$ von Relevanz w�ren, nicht aber der Druck in Zelle $i$ selbst (analog f�r die y-Richtung und die $y$-Richtung). Die Verwendung eines versetzten Gitters bringt eine Abhilfe f�r dieses Problem. Es gibt nun 2 weitere Gitter f�r beide Geschwindigkeitskomponenten. F�r $u$ wird das kartesische Gitter f�r $\Phi$  um eine halbe Zellbreite in horizontaler Richtung verschoben, d.h. die neuen horizontalen Kanten f�r das Gitter liegen auf den alten Kanten, die vertikalen Kanten gehen nun aber durch die Zellmittelpunkte der physikalischen Zellen f�r $\Phi$. Analoges gilt f�r das Gitter f�r $v$. Dort laufen nun die horizontalen Kanten durch die Zellmittelpunkte der physikalischen Zellen f�r $\Phi$. Diese Betrachtungsweise ist ebenfalls sinnvoll, da die Geschwindigkeitskomponenten f�r die Berechnung der Fl�sse �ber die Zellkanten gerade auf diesen definiert sind, und nicht in den Zellmittelpunkte der physikalischen Zellen f�r $\Phi$. \\
Es wird die Zellkante an der Stelle $i+\frac{1}{2},j$ betrachtet. Unter Verwendung zentraler Differenz zur Approximation des Druckgradienten in \eqref{eq:impulsglg_2D} bzw. \eqref{eq:impulsglg_1D}, sprich
\begin{equation} \label{eq:druck_approx_x}
\dfrac{\partial p}{\partial x}\approx \dfrac{p_{i+1,j}-p_{i,j}}{x_{i+1}-x_{i}}
\end{equation}
f�r das $u$-Gitter und 
\begin{equation} \label{eq:druck_approx_y}
\dfrac{\partial p}{\partial y}\approx \dfrac{p_{i,j+1}-p_{i,j}}{y_{j+1}-y_{i}}
\end{equation}
f�r das $v$-Gitter, gelangt man (analog zu Formeln in Kapitel 3)\textcolor{red}{VERWEIS?! JA NEIN?!} zu folgender Diskreitisierung:
\begin{align} \label{eq:diskretisierung_u}
\tilde{a}_{i+\frac{1}{2},j}\cdot u_{i+\frac{1}{2},j}^{k}= \notag \\
& a_{i+\frac{3}{2},j}\cdot u_{i+\frac{3}{2},j}^{k}+a_{i-\frac{1}{2},j}\cdot u_{i-\frac{1}{2},j}^{k}+a_{i+\frac{1}{2},j+1}\cdot u_{i+\frac{1}{2},j+1}^{k}+a_{i+\frac{1}{2},j-1}\cdot u_{i+\frac{1}{2},j-1}^{k}+ \notag \\
& b+ \dfrac{p_{i,j}-p_{i+1,j}}{x_{i+1}-x_{i}}\cdot\left(x_{i+1}-x_{i}\right)\Delta y_{j},
\end{align}
bzw.
\begin{align} \label{eq:diskretisierung_v}
\tilde{a}_{i,j+\frac{1}{2}}\cdot u_{i,j+\frac{1}{2}}^{k}= \notag \\
& a_{i+1,j+\frac{1}{2}}\cdot u_{i+1,j+\frac{1}{2}}^{k}+a_{i-1,j+\frac{1}{2}}\cdot u_{i-1,j+\frac{1}{2}}^{k}+a_{i,j+\frac{3}{2}}\cdot u_{i,j+\frac{3}{2}}^{k}+a_{i,j-\frac{1}{2}}\cdot u_{ii,j-\frac{1}{2}1}^{k}+ \notag \\
& b+\dfrac{p_{i,j}-p_{i,j+1}}{x_{i+1}-x_{i}}\cdot\left(y_{j+1}-y_{j}\right)\Delta x_{i}.
\end{align}
In den Formeln wurden die Indizes jeweils auf das physikalische Gitter des Skalars $\Phi$ bezogen, da dies in der Software genau so umgesetzt worden ist. Die Formeln f�r die Koeffizienten entsprechen denen aus Kapitel 3\textcolor{red}{VERWEIS?! JA NEIN?!} auf dem oben definieirten versetztem Gitter.

\section{Validierung}
\subsection{Couette-Str�mung}

\subsection{Poiseuille-Str�mung}